% Options for packages loaded elsewhere
\PassOptionsToPackage{unicode}{hyperref}
\PassOptionsToPackage{hyphens}{url}
\PassOptionsToPackage{dvipsnames,svgnames,x11names}{xcolor}
%

\documentclass[
  letterpaper,
]{trb}

\usepackage{fullpage}
\usepackage[pagewise]{lineno}
\linenumbers
\usepackage{newtxtext}


\makeatletter
\newcounter{wordcounter}
\setcounter{wordcounter}{6513}

\newcounter{tablecounter}
\setcounter{tablecounter}{3}

\newcounter{totalwordcounter}
\newcommand{\totalwordcount}{%
  \setcounter{totalwordcounter}{6513}
  \addtocounter{totalwordcounter}{\numexpr250*3}
  \number\value{totalwordcounter}% Output the number
	\renewcommand{\totalwordcount}{\number\value{totalwordcounter}}
}
\makeatother

\usepackage{amsmath,amssymb}
\usepackage{iftex}
\ifPDFTeX
  \usepackage[T1]{fontenc}
  \usepackage[utf8]{inputenc}
  \usepackage{textcomp} % provide euro and other symbols
\else % if luatex or xetex
  \usepackage{unicode-math}
  \defaultfontfeatures{Scale=MatchLowercase}
  \defaultfontfeatures[\rmfamily]{Ligatures=TeX,Scale=1}
\fi
\usepackage{lmodern}
\ifPDFTeX\else  
    % xetex/luatex font selection
\fi
% Use upquote if available, for straight quotes in verbatim environments
\IfFileExists{upquote.sty}{\usepackage{upquote}}{}
\IfFileExists{microtype.sty}{% use microtype if available
  \usepackage[]{microtype}
  \UseMicrotypeSet[protrusion]{basicmath} % disable protrusion for tt fonts
}{}
\makeatletter
\@ifundefined{KOMAClassName}{% if non-KOMA class
  \IfFileExists{parskip.sty}{%
    \usepackage{parskip}
  }{% else
    \setlength{\parindent}{0pt}
    \setlength{\parskip}{6pt plus 2pt minus 1pt}}
}{% if KOMA class
  \KOMAoptions{parskip=half}}
\makeatother
\usepackage{xcolor}
\setlength{\emergencystretch}{3em} % prevent overfull lines
\setcounter{secnumdepth}{5}
% Make \paragraph and \subparagraph free-standing
\ifx\paragraph\undefined\else
  \let\oldparagraph\paragraph
  \renewcommand{\paragraph}[1]{\oldparagraph{#1}\mbox{}}
\fi
\ifx\subparagraph\undefined\else
  \let\oldsubparagraph\subparagraph
  \renewcommand{\subparagraph}[1]{\oldsubparagraph{#1}\mbox{}}
\fi


\providecommand{\tightlist}{%
  \setlength{\itemsep}{0pt}\setlength{\parskip}{0pt}}\usepackage{longtable,booktabs,array}
\usepackage{calc} % for calculating minipage widths
% Correct order of tables after \paragraph or \subparagraph
\usepackage{etoolbox}
\makeatletter
\patchcmd\longtable{\par}{\if@noskipsec\mbox{}\fi\par}{}{}
\makeatother
% Allow footnotes in longtable head/foot
\IfFileExists{footnotehyper.sty}{\usepackage{footnotehyper}}{\usepackage{footnote}}
\makesavenoteenv{longtable}
\usepackage{graphicx}
\makeatletter
\def\maxwidth{\ifdim\Gin@nat@width>\linewidth\linewidth\else\Gin@nat@width\fi}
\def\maxheight{\ifdim\Gin@nat@height>\textheight\textheight\else\Gin@nat@height\fi}
\makeatother
% Scale images if necessary, so that they will not overflow the page
% margins by default, and it is still possible to overwrite the defaults
% using explicit options in \includegraphics[width, height, ...]{}
\setkeys{Gin}{width=\maxwidth,height=\maxheight,keepaspectratio}
% Set default figure placement to htbp
\makeatletter
\def\fps@figure{htbp}
\makeatother
\newlength{\cslhangindent}
\setlength{\cslhangindent}{1.5em}
\newlength{\csllabelwidth}
\setlength{\csllabelwidth}{3em}
\newlength{\cslentryspacingunit} % times entry-spacing
\setlength{\cslentryspacingunit}{\parskip}
\newenvironment{CSLReferences}[2] % #1 hanging-ident, #2 entry spacing
 {% don't indent paragraphs
  \setlength{\parindent}{0pt}
  % turn on hanging indent if param 1 is 1
  \ifodd #1
  \let\oldpar\par
  \def\par{\hangindent=\cslhangindent\oldpar}
  \fi
  % set entry spacing
  \setlength{\parskip}{#2\cslentryspacingunit}
 }%
 {}
\usepackage{calc}
\newcommand{\CSLBlock}[1]{#1\hfill\break}
\newcommand{\CSLLeftMargin}[1]{\parbox[t]{\csllabelwidth}{#1}}
\newcommand{\CSLRightInline}[1]{\parbox[t]{\linewidth - \csllabelwidth}{#1}\break}
\newcommand{\CSLIndent}[1]{\hspace{\cslhangindent}#1}

\usepackage{booktabs}
\usepackage{longtable}
\usepackage{array}
\usepackage{multirow}
\usepackage{wrapfig}
\usepackage{float}
\usepackage{colortbl}
\usepackage{pdflscape}
\usepackage{tabu}
\usepackage{threeparttable}
\usepackage{threeparttablex}
\usepackage[normalem]{ulem}
\usepackage{makecell}
\usepackage{xcolor}
\usepackage{rotating}
\usepackage{orcidlink}
\definecolor{mypink}{RGB}{219, 48, 122}
\makeatletter
\makeatother
\makeatletter
\@ifpackageloaded{bookmark}{}{\usepackage{bookmark}}
\makeatother
\makeatletter
\@ifpackageloaded{caption}{}{\usepackage{caption}}
\AtBeginDocument{%
\ifdefined\contentsname
  \renewcommand*\contentsname{Table of contents}
\else
  \newcommand\contentsname{Table of contents}
\fi
\ifdefined\listfigurename
  \renewcommand*\listfigurename{List of Figures}
\else
  \newcommand\listfigurename{List of Figures}
\fi
\ifdefined\listtablename
  \renewcommand*\listtablename{List of Tables}
\else
  \newcommand\listtablename{List of Tables}
\fi
\ifdefined\figurename
  \renewcommand*\figurename{Figure}
\else
  \newcommand\figurename{Figure}
\fi
\ifdefined\tablename
  \renewcommand*\tablename{Table}
\else
  \newcommand\tablename{Table}
\fi
}
\@ifpackageloaded{float}{}{\usepackage{float}}
\floatstyle{ruled}
\@ifundefined{c@chapter}{\newfloat{codelisting}{h}{lop}}{\newfloat{codelisting}{h}{lop}[chapter]}
\floatname{codelisting}{Listing}
\newcommand*\listoflistings{\listof{codelisting}{List of Listings}}
\makeatother
\makeatletter
\@ifpackageloaded{caption}{}{\usepackage{caption}}
\@ifpackageloaded{subcaption}{}{\usepackage{subcaption}}
\makeatother
\makeatletter
\@ifpackageloaded{tcolorbox}{}{\usepackage[skins,breakable]{tcolorbox}}
\makeatother
\makeatletter
\@ifundefined{shadecolor}{\definecolor{shadecolor}{rgb}{.97, .97, .97}}
\makeatother
\makeatletter
\makeatother
\makeatletter
\makeatother
\ifLuaTeX
  \usepackage{selnolig}  % disable illegal ligatures
\fi
\IfFileExists{bookmark.sty}{\usepackage{bookmark}}{\usepackage{hyperref}}
\IfFileExists{xurl.sty}{\usepackage{xurl}}{} % add URL line breaks if available
\urlstyle{same} % disable monospaced font for URLs
\hypersetup{
  pdftitle={Evaluating the Impacts of Parameter Uncertainty in Transportation Demand Models},
  pdfauthor={Gregory S. Macfarlane; Natalie Mae Gray},
  colorlinks=true,
  linkcolor={blue},
  filecolor={Maroon},
  citecolor={Blue},
  urlcolor={red},
  pdfcreator={LaTeX via pandoc}}


\title{Evaluating the Impacts of Parameter Uncertainty in Transportation
Demand Models}
\author{
Gregory S. Macfarlane\\Brigham Young University\\Civil and Construction
Engineering\\\href{mailto:gregmacfarlane@byu.edu}{gregmacfarlane@byu.edu}\\ORCID: 0000-0003-3999-7584\\ \\ 
Natalie Mae Gray\\Brigham Young
University\\\\\href{mailto:nmgray@byu.edu}{nmgray@byu.edu}\\}
\date{2023-07-03}
\begin{document}
\maketitle
\newpage
\begin{abstract}
The inherent uncertainty in travel forecasting models --- arising from
potential and unkown errors in input data, parameter estimation, or
model formulation --- is receiving increasing attention from the
scholarly and practicing community. In this research, we investigate the
variance in forecasted traffic volumes resulting from varying the mode
and destination choice parameters in an advanced trip-based travel
demand model. Using Latin hypercube sampling to construct several
hundred combinations of parameters across the plausible parameter space,
we introduce substantial changes to implied impedances and modal
utilities. However, the aggregate effects of of these changes on
forecasted traffic volumes is small, with a variance of approximately 1
percent on high-volume facilities. Thus, parameter uncertainty does not
appear to be a significant factor in forecasting traffic volumes using
transportation demand models.
\end{abstract}
\newpage
\ifdefined\Shaded\renewenvironment{Shaded}{\begin{tcolorbox}[boxrule=0pt, frame hidden, enhanced, interior hidden, borderline west={3pt}{0pt}{shadecolor}, breakable, sharp corners]}{\end{tcolorbox}}\fi

\bookmarksetup{startatroot}

\hypertarget{introduction}{%
\section{Introduction}\label{introduction}}

The inherent accuracy and uncertainty in travel forecasting models is
receiving increasing attention from the scholarly and practicing
community. Given that such models are used in the allocation of billions
of dollars of infrastructure financing each year, the financial risks
for inaccurate or imprecise forecasts are high (\emph{1}, \emph{2}).

Transportation demand forecasting models, like other
mathematical-statistical models, might be abstracted to the following
basic form,

\[
y = f(X, \beta)
\]where \(y\) is the variable being predicted based on input data \(X\),
moderated through a specific functional form \(f()\) and parameters
\(\beta\). Three general sources of error may lead a forecast value
\(\hat{y}\) to differ from the ``true'' or ``actual'' value of \(y\)
(\emph{3}):

\begin{enumerate}
\def\labelenumi{\arabic{enumi}.}
\tightlist
\item
  The input data \(X\) might contain errors, due to inaccuracies in the
  base year, or an inaccurate projection of land use, petroleum price,
  or other key input variable. This was among the primary issues
  identified by (\emph{4}) in a historical analysis of the accuracy of
  travel forecasts.
\item
  The model form \(f()\) may be improperly specified. Variables that
  play a major role in travel behavior may not be included due to lack
  of information, or the unobserved error components may have a
  different correlation than was assumed during model development. A
  detailed description of specifying mode choice model variables and
  nesting of error structures is given by (\emph{5}).
\item
  The parameter estimates \(\hat{\beta}\) of the ``true'' parameters
  \(\beta\) may have incorrect values. This may be because the
  parameters were estimated on an improperly specified model \(f()\), or
  because the estimation dataset was improperly weighted.
\end{enumerate}

Of these potential sources of error, only the third is substantively
addressed in classical statistics. The standard errors of the model
parameter estimates in a theoretical perspective address the parameter
uncertainty question to a great degree. Yet even this source of
uncertainty has been largely ignored in transportation forecasts, and
model development documentation often elides the variance in these
values completely (\emph{6}). (\emph{7}) examined the effects of this
parameter uncertainty in a trip-based model of a contrived 25 zone
region, but a systemic analysis of this uncertainty in a practical model
is not common.

In this research, we investigate the uncertainty in traffic forecasts
resulting from plausible parameter uncertainty in an advanced trip-based
transportation demand model. Using a Latin hypercube sampling (LHS)
methodology, we simulate one hundred potential parameter sets for a
combined mode and destination choice model in Roanoke, Virginia, USA. We
then assign the resulting trip matrices to the highway network for the
region and evaluate the PM and daily assigned traffic volumes alongside
the variation in implied impedance and accessibility.

This paper proceeds first with a description of the model design and
simulation sampling methodology in Chapter~\ref{sec-methods}, followed
by a discussion of the variation in mode, destination, and traffic
performance measures in Chapter~\ref{sec-results}. The paper concludes
in Chapter~\ref{sec-conclusions} with a summary of the key findings
alongside a presentation of limitations and related indications for
future research.

\bookmarksetup{startatroot}

\hypertarget{literature-review}{%
\section{Literature Review}\label{literature-review}}

Uncertainty has been examined in various ways over the last two decades,
and is becoming increasingly important for researchers. This review
looks at why uncertainty is important to evaluate in transportation
demand models, and research that has been done to evaluate uncertainty.
(\emph{3}) has an extensive literature review on this topic. An overview
of the literature and which source of uncertainty they evaluate can be
found in Table~\ref{tbl-authors}.

\hypertarget{tbl-authors}{}
\begin{longtable}[]{@{}
  >{\raggedright\arraybackslash}p{(\columnwidth - 2\tabcolsep) * \real{0.5286}}
  >{\raggedright\arraybackslash}p{(\columnwidth - 2\tabcolsep) * \real{0.4714}}@{}}
\caption{\label{tbl-authors}Studies of Forecasting
Uncertainty}\tabularnewline
\toprule\noalign{}
\begin{minipage}[b]{\linewidth}\raggedright
Reference
\end{minipage} & \begin{minipage}[b]{\linewidth}\raggedright
Uncertainty Source(s) Evaluated
\end{minipage} \\
\midrule\noalign{}
\endfirsthead
\toprule\noalign{}
\begin{minipage}[b]{\linewidth}\raggedright
Reference
\end{minipage} & \begin{minipage}[b]{\linewidth}\raggedright
Uncertainty Source(s) Evaluated
\end{minipage} \\
\midrule\noalign{}
\endhead
\bottomrule\noalign{}
\endlastfoot
Rodier et al. (\emph{8}) & Input Data \\
Zhao and Kockelman (\emph{7}) & Input Data \& Parameter Estimates \\
Clay and Jonson (\emph{9}) & Input Data \& Parameter Estimates \\
Flyvbjerg et al. (\emph{1}) & Model Form \\
Armoogum et al. (\emph{10}) & Model Form \\
Duthie et al. (\emph{11}) & Input Data \& Parameter Estimates \\
Welde and Odeck (\emph{12}) & Model Form \\
Yang et al. (\emph{13}) & Input Data \& Parameter Estimates \\
Manzo et al. (\emph{14}) & Input Data \& Parameter Estimates \\
Petrik et al. (\emph{15}) & Input Data \& Parameter Estimates \\
Petrik et al. (\emph{16}) & Model Form \& Parameter Estimates \\
Hoque et al. (\emph{4}) & Input Data \\
\end{longtable}

Model accuracy is the basis for why uncertainty of input data and/or
parameter estimates are important to study. Travel forecasters have
always been cognizant of the uncertainty in their forecasts, especially
as project decisions are made using these models, often with high
financial impacts. (\emph{1}), (\emph{10}), and (\emph{12}) looked at
forecasting uncertainty in transportation models.

(\emph{1}) collected data from various forecasting traffic models with
an emphasis on rail projects. They used the forecast data for a given
year and the actual value that was collected for the same year. Their
study found that there is a statistical significance in the difference
of the estimated and actual values. Rail projects are generally
overestimating passenger forecasts by 106\%, and half of road projects
have a traffic forecast difference of plus or minus 20\%. They did not
identify where this inaccuracy came from, but they identified that it
was important for future research.

(\emph{10}) looked at uncertainty within a forecasting model for the
Paris and Montreal metropolitan regions. The sources of uncertainty
analysed were calibration of the model, behavior of future generations,
and demographic projections. A jackknife technique, rather than sampling
methods, was used to estimated confidence intervals for each source of
error using multiple years of analysis. This technique is a way to
reduce the bias of an estimator and permits the estimation of confidence
intervals to produce variance estimates. They found that the longer the
forecasting period was, the larger the uncertainty. Generally the model
forecasted within 10-15\%, reaching higher percentage ranges for
variables with small values or small sample sizes.

(\emph{12}) compared actual and forecasted traffic values for 25 toll
and 25 toll free roads in Norway. They evaluated the accuracy of
Norwegian transportation planning models over the years. Generally
traffic models overestimate traffic. This study found that toll
projects, on average, overestimated traffic, but only by an average of
2.5\%. Toll free projects, however, underestimated traffic by an average
of 19\%. They concluded that Norwegian toll projects have been fairly
accurate, with a probable cause coming from the scrutiny that planners
get when developing a toll project. A similar scrutiny should then also
be placed on toll free projects as they are significantly less accurate.

These articles show that models have errors which effects traffic
projections by a significant amount. These articles identified that
error existed but did not quantitatively identify the source of the
error. The most researched error source has been on model form but that
research has mostly been excluded in this review as it is not the main
focus of this research. The second most researched form has been on
input data. Chronologically, (\emph{8}), (\emph{7}), (\emph{9}),
(\emph{11}), (\emph{13}), (\emph{14}), and (\emph{15}) have all
researched input error, with all but the first also looking at parameter
estimate error as well. Parameter estimation error has been the least
researched source of uncertainty, where there have been no studies
focused only on that source of error. (\emph{16}) looked at parameter
estimates, but with a focus also on model form error. The details of
each study are described below in chronological order.

(\emph{8}) looked at uncertainty in socioeconomic projections
(population and employment, household income, and petroleum prices) at
the county-level for the Sacramento, California region. They wanted to
know if the uncertainty in the range of plausible socioeconomic values
was a significant source of error in the projection of future travel
patterns and vehicle emissions. They identified ranges for population
and employment, household income, and petroleum price for two scenario
years (2005 and 2015). The ranges varied based on the scenario year and
the socioeconomic variable. They changed one variable at a time for a
total of 19 iterations of the model run for 2005 and 21 iterations for
2015. Their results indicated that the error in projections for
household income and petroleum prices is not a significant source of
uncertainty, but error ranges for population and employment projections
are a significant source for changes in travel and emissions. The input
data of population and employment were a significant factor to the model
result uncertainty.

(\emph{7}) looked at the propagation of uncertainty through each step of
a trip-based travel model from variation among inputs and parameters.
This analysis used a traditional four-step urban transportation planning
process (trip generation, trip attraction, mode split, and trip
assignment) on a 25-zone sub-model of the Dallas-Fort Worth metropolitan
region. Monte Carlo simulation was used to vary the input and parameter
values. These values were all ranged using a coefficient of variation
(\(c_v\)) of 0.30. The four-step model was run 100 times with 100
different sets of input and parameter values. The results of these runs
showed that uncertainty increased in the first three steps of the model
and the final assignment step reduced the compounded uncertainty,
although not below the levels of input uncertainty. The authors
determined that uncertainty propagation was significant from changes in
inputs and parameters, but the final step nearly stabilizes the
uncertainty to the same amount as assumed (0.30 \(c_v\) assumption with
a 0.31 \(c_v\) in the results of trip assignment).

Another study that looked at input data uncertainty was (\emph{9}).
These researchers varied three inputs and one parameter to analyze
uncertainty of outputs on a fully integrated land use and travel demand
model of six counties in the Sacramento, California region. The
variables used for analysis were productions, commercial trip generation
rates, perceived out-of-pocket costs of travel for single occupant
vehicles, and concentration parameter. Exogenous production, commercial
trip generation rates, and the concentration parameter were varied by
plus or minus 10, 25 and 50\%, while the cash cost of driving was varied
by plus or minus 50 and 100\%. This resulted in 23 model runs, one for
each changed variable and one for the base scenario. Their research
found that any uncertainty in the inputs resulted in large difference in
the vehicle miles traveled output, although this difference was a lower
percentage than the uncertainty in the input.

(\emph{11}) evaluated uncertainty at a different level. They use a small
generic gravity-based land use model with the traditional four steps,
using a coefficient of variation of 0.3 from (\emph{7}) for input and
parameters, although using antithetic sampling. In this sampling method,
pairs of negatively correlated realizations of the uncertain parameters
are used to obtain an estimate of the expected value of the function.
The uncertainty was evaluated on the rankings of various transportation
improvement projects. They found that there are a few significant
differences that arise when changing the input and parameter values that
result in different project rankings, and thus neglecting uncertainty
can lead to suboptimal network improvement decisions.

(\emph{13}) evaluated a quantitative uncertainty analysis of a combined
travel demand model. They looked at input and parameter uncertainty
\emph{also} using a coefficient of variation of 0.30. Rather than using
a random sampling method for choices they used a systematic framework
with a variance-covariance matrix. Their research found that the
coefficient of variation of the outputs are similar to the coefficient
of variation of the inputs, and that the effect of parameter uncertainty
on output uncertainty is generally higher than that of input
uncertainty. This finding contradicts the finding of (\emph{7}). The
authors concluded that improving the accuracy of parameter estimation is
more effective that that of improving input estimation as they found
that in most steps of the model, the impact of parameter uncertainty was
more important that that of input uncertainty.

(\emph{14}) looked at uncertainty on model input and parameters for a
trip-based transportation demand model in a small Danish town. They used
a triangular distribution with LHS to create the range in parameters,
and using the information from (\emph{7}) they also used a coefficient
of variation of 0.30 and 100 draws, choosing these values at they had
been previously used. Their addition to the research of uncertainty, was
by examining uncertainty under different levels of congestion. Their
research found that there is an impact on the model output from the
change in input and parameter uncertainty and requires attention when
planning. Also, model output uncertainty was not sensitive to the level
of congestion.

(\emph{15}) evaluated uncertainty in mode shift predictions due to
uncertainty from input parameters, socioeconomic data, and alternative
specific constants. This study was based on a high-speed rail project in
Portugal as a component of the Trans-European Transport Network. They
collected survey data and developed discrete choice models. The authors
created their own parameter values from the collected data, obtaining
the mean or ``best'' value from the surveys and the corresponding
t-statistic. With these they generated 10,000 samples each of parameter
values, socioeconomic inputs, and mode-specific constants, using
bootstrap re-sampling, Monte Carlo sampling, and triangular distribution
methods respectively. The authors found that variance in alternative
specific attributes is the major contributor to output uncertainty in
comparison to parameter variance or socioeconomic variance.
Socioeconomic data had the least contribution to overall output
variance, and there was a relatively insignificant mode shift due to
variability in parameters.

(\emph{16}) used an activity based microsimulation travel demand model
for Singapore to evaluate model form and parameter uncertainty. This
model has 22 sub-models and 817 parameters. The authors determined which
of the 817 parameters the sub-models were most sensitive to and applied
a full sensitivity analysis of the top 100 of the parameters, preserving
correlations. Using the mean parameter value and the standard deviations
they had for all of them they used Latin hypercube sampling with 100
draws to look at the outcomes of the change in each parameter value.
Different sized samples of the model population were also considered in
their research. They found that of the 100 most sensitive parameter
values, the outcome coefficient of variation varied from 3\% to 49\%.
The variance of the parameter variables did not exceed 19\%, and thus
the results from the parameter uncertainty were higher than the variance
in the parameters. They also found that the results of the parameter
uncertainty was higher than simulation uncertainty.

In transportation demand models, when uncertainty is analysed, most
error research to this point has focused on input uncertainty or model
forms, rather than parameter estimate uncertainty (\emph{3}). In this
literature review 12 articles have been included. Of those 12, two look
at input data as the only focus of their uncertainty research, three
focus on model form uncertainty, one looks at both model form and
parameter estimate uncertainty, and six focus on both input data and
parameter estimate uncertainty. No researchers have looked at parameter
estimate uncertainty as the only source of error in their models. This
is a gap in uncertainty research and is why that for this study, the
parameter estimates within the model uncertainty is of the most
immediate concern. When parameter uncertainty has been examined in
literature it is often in conjunction with input errors, or on
relatively small and impractical models. No researchers have used real
models for their analyses. There have also been some conflicting
conclusions on how parameter variance impacts output uncertainty. There
is a research need within uncertainty analyses to have additional and
repetitive studies that can confirm the outcome of previous research,
since it is a relatively new topic of discussion. Uncertainty research
is needed as transportation demand models provide estimates and
forecasts for decision and policy makers. An inaccurate model or large
output variance could change what decisions are made and when
(\emph{17}). There is also a research need within uncertainty for both
activity-based models, and destination and transport mode choice model
components. In this research, I investigated the variance in forecasted
traffic volumes resulting from varying the mode and destination choice
parameters in an advanced trip-based travel demand model, using combined
mode and destination choice models, coupled with highway assignment.

\bookmarksetup{startatroot}

\hypertarget{sec-methods}{%
\section{Model Design and Methodology}\label{sec-methods}}

This section describes the process by which a transportation demand
model has been created and developed for the purposes of this
uncertainty analysis. A method for how to evaluate uncertainty was also
described and a decision is made for how to use them to evaluate model
uncertainty.

\hypertarget{model-design}{%
\subsection{Model Design}\label{model-design}}

To examine the effects of parameter input sensitivity, I developed a
trip-based travel demand model. The model was developed from the Roanoke
Valley Transportation Planning Organization
\href{https://github.com/xinwangvdot/rvtpo}{(RVTPO)} Model. The RVTPO
model provides an ideal testing environment for this research because it
uses an integrated mode and destination choice framework common in more
advanced trip-based models. At the same time, its small size
(approximately 215 zones) means the entire model runs in a few minutes
allowing for efficient testing of multiple model runs.

The total passenger trips \(T\) traveling from zone \(i\) to zone \(j\)
on the highway in a period \(t\) is
\begin{equation}\protect\hypertarget{eq-trips}{}{
T_{ijt} = P_i * \mathcal{P}_{\mathrm{auto}}(\beta, C_{ijt}) * \mathcal{P}_j(\gamma, A_j, MCLS_{ijt}) * \Delta_t 
}\label{eq-trips}\end{equation} where \(P\) is the productions at zone
\(i\); \(\mathcal{P}_{\mathrm{car}}\) is the car mode choice probability
determined by utility parameters \(\beta\) and the travel costs \(C\)
between \(i\) and \(j\) at time period \(t\); \(\mathcal{P}_{j}\) is the
destination choice probability of choosing destination \(j\) given the
utility parameters \(\gamma\), attractions \(A\), and the impedance as
the mode choice model logsum \(MCLS_{ijt}\). A time-of-day and direction
factor \(\Delta\) finalizes the total assigned trips.

The productions \(P_i\), and attractions \(A_j\) were extracted from the
RVTPO Model and held constant. The attractions are determined from the
socioeconomic (SE) data. The SE data included information by TAZ for the
total population, number of households, total workers, and workers by
employment type. The trip productions are organized by TAZ and trip
purpose. The trip purposes used in this model are Home Based Work (HBW),
Home Based Other (HBO), Non-Home Based (NHB), Commercial Vehicles (CV),
Internal-External (IXXI), and External-External (XX). Only the first
three are analysed, but all of the purposes are assigned to the network.
CV, IXXI, and XX trips were kept fixed for this analysis.

The two parameter vectors \(\beta\) and \(\gamma\) describe the mode
choice model and destination choice model coefficients, respectively.
Mode choice estimates how many trips from \(i\) to \(j\) will happen on
each available mode \(k\).This model analyses three modes of
transportation: auto, non-motorized, and transit. The mode by which a
trip is made is determined by calculated utilities for the three modes.
These utilities take inputs from parameter values and time and distance
skims \(X\). Skims are either the time or distance to travel between
zone pairs. Travel time for auto used the single occupancy vehicle peak
time, non-motorized travel time used the distance skim multiplied by a
factor of average walking speed (3 mph), and transit time used the walk
to bus peak time. The mode choice parameters (constants and
coefficients) were also obtained from the RVTPO model. These values are
shown in Table~\ref{tbl-choicecoeff}.

\hypertarget{tbl-choicecoeff}{}
\begin{table}
\caption{\label{tbl-choicecoeff}Choice model parameters }\tabularnewline

\centering
\begin{tabular}[t]{llrrr}
\toprule
 & Variable & HBW & HBO & NHB\\
\midrule
\addlinespace[0.3em]
\multicolumn{5}{l}{\textbf{Mode Choice Coefficients}}\\
\hspace{1em}In-vehicle travel time & $\beta_{ivtt}$ & -0.0250 & -0.0150 & -0.0200\\
\hspace{1em}Travel cost & $\beta_{tc}$ & -0.0016 & -0.0024 & -0.0025\\
\hspace{1em}Walk distance & $\beta_{wd}$ & -0.0625 & -0.0375 & -0.0500\\
\hspace{1em}Auto operating cost (cents/mile) & $\beta_{ac}$ & 13.6000 & 13.6000 & 13.6000\\
\addlinespace[0.3em]
\multicolumn{5}{l}{\textbf{Mode Choice Constants}}\\
\hspace{1em}Transit constant & $k_{trn}$ & -0.3903 & -1.9811 & -2.2714\\
\hspace{1em}NonMotorized constant & $k_{nmot}$ & -1.2258 & -0.3834 & -0.8655\\
\addlinespace[0.3em]
\multicolumn{5}{l}{\textbf{Destination Choice Parameters}}\\
\hspace{1em}Households & $\gamma_{hh}$ & 0.0000 & 1.1657 & 0.5664\\
\hspace{1em}Other + Office & $\gamma_{oth + off}$ & 0.0000 & 0.8064 & 0.5626\\
\hspace{1em}Office & $\gamma_{off}$ & 0.4586 & 0.0000 & 0.0000\\
\hspace{1em}Other & $\gamma_{oth}$ & 1.6827 & 0.0000 & 0.0000\\
\hspace{1em}Retail & $\gamma_{ret}$ & 0.6087 & 2.2551 & 5.1190\\
\bottomrule
\end{tabular}
\end{table}

The utility equations for the mode choice model are as follows: \[
\begin{aligned}
U_{auto} &= \beta_{ivtt} * X_{auto} + \beta_{tc} * \beta_{ac} * X_{dist}\\
U_{nmot} &= k_{nmot} + 20 * \beta_{wd}*X_{nmot}\\
U_{trn} &= k_{trn} + \beta_{ivtt} * X_{trn}
\end{aligned}
\] These utilities are used to calculate the MCLS by:
\begin{equation}\protect\hypertarget{eq-mcls}{}{
MCLS_{ij} = \ln\left(\sum_{k \in K} e^{U_{ijk}}\right).
}\label{eq-mcls}\end{equation} If the distance was greater than 2 miles,
non-motorized travel was excluded.

This logsum value is then used as the primary impedance for a
destination choice model (\emph{18}). Destination choice estimates
travel patterns based on mode choice, trip generators (workers and
households), and destination choice parameters. These parameter values
are also shown in Table~\ref{tbl-choicecoeff}. The destination choice
utility is the primary impedance (mode choice logsum value) plus the
natural log of the size term, where the sized term is calculated as:
\begin{equation}\protect\hypertarget{eq-dcsizeterm}{}{
A_j = \gamma_{hh} * \mathrm{HH} + \gamma_{off} * \mathrm{OFF} + \gamma_{ret} * \mathrm{RET} + \gamma_{oth} * \mathrm{OTH} + \\ \gamma_{oth+off} * \mathrm{OFFOTH}
}\label{eq-dcsizeterm}\end{equation} HH is the total households in zone
\(j\). OFF, RET, and OTH are the jobs in zone \(j\) by employment type
office, retail, and other respectively. The destination choice utility
is then transformed into a destination choice logsum value with:
\begin{equation}\protect\hypertarget{eq-dcls}{}{
DCLS = \ln \left(\sum_{j \in J} e^{\ln(A_j) + 1* MCLS_{ij}}\right)
}\label{eq-dcls}\end{equation}

The probability of both the mode choice and destination choice are
calculated using the exponentiated utility divided by the corresponding
logsum. These probabilities in conjunction with the trip productions can
calculate the number of production-attraction (PA) trips between each
zone by each mode and purpose. The auto trips are calculated by
multiplying the probability of the destination by PA pair, the
productions for each origin, and the probability of an auto mode choice
by PA pair. This results in PA auto trips. The same process is followed
for the other two modes. These PA trips are converted into origin
destination (OD) trips by multiplying the trips by corresponding time of
day factors (see \#eq-trips). These trips are calculated using Bentley's
CUBE and the RVTPO model. The trips, by time period, are assigned to the
highway network by the shortest path by time using free flow speed and
with link capacity as a restriction.

\hypertarget{uncertainty-design}{%
\subsection{Uncertainty Design}\label{uncertainty-design}}

Within the mode and destination models there exists uncertainty within
the parameters in Table~\ref{tbl-choicecoeff}. Sampling methods can take
the defined uncertainty and choose potential parameter values within the
possible range. Two common methods for parameter sampling include, Monte
Carlo (MC) simulation and Latin hypercube sampling (LHS). MC simulation
draws independently from multiple distributions, while LHS makes draws
that cover the parameter space more efficiently and can capture the
joint distribution between two or more parameter values (\emph{19}). As
a result, LHS can reduce the number of draws needed to fully re-create
the statistical variance in a model, but the amount of reduction is
unknown and may not be universal to all problems (\emph{13}).

With the trip-based model described above, MC and LHS methods were used
to develop alternative parameter sets to evaluate uncertainty. To
identify a standard deviation for each parameter, a coefficient of
variation was used. A set coefficient of variation of 0.10 was used for
the four mode choice coefficients and the destination choice parameters.
The mode choice constants were kept the same across all iterations.
Literature had identified a coefficient of variation of 0.30, but for
this analysis that caused an unrealistic value of time, and thus it was
changed to be 0.10 (\emph{7}). Value of time is a ratio in units of
money per time that should be compared to the regional wage rate. Using
a \(c_v\) of 0.30 the value of time range was from \(\$2\) to \(\$32\)
/hr, whereas using a \(c_v\) of 0.10 the range was \(\$6\) to \(\$14\)
/hr. The latter seemed more rational because it is related to wage rates
and thus a \(c_v\) of 0.10 was used for our analysis. The standard
deviation was equal to 0.10 multiplied by the mean, where the mean
values in this situation are the base scenario parameters (as identified
in Table~\ref{tbl-choicecoeff} ).

The MC random sampling uses the R function of \texttt{rnorm}. LHS uses
the \texttt{lhs} package in R. Since this package only chooses variables
on a zero to one scale, the values given use a function to put the
random sampling on the right scale needed for the given parameter. The
full code for both methods can be found in a public
\href{https://github.com/natmaegray/sensitivity_thesis}{GitHub
repository}. One hundred and 600 draws of random samples for both
methods are generated. With these generated parameters, the mode choice
model step was run for every set of input parameters for each purpose.
The average MCLS value for each run was determined to compare each
continuous draw. This allowed us to see how many iterations of which
sampling type would be sufficient to show a full range of possible
outcomes.

The parameters generated were compared for both sampling methods.
Figure~\ref{fig-parameter} shows the distributions for the HBW
parameters when using 100 and 600 draws. These distributions show that
LHS gives normally distributed parameters with fewer draws than MC
sampling: at 100 draws LHS shows a nearly perfect normal distribution,
where there are some discrepancies for the MC generated parameters.
These Figures show that LHS is likely to estimate the full variance of
the results with fewer draws.

\begin{figure}

\begin{minipage}[t]{0.50\linewidth}

{\centering 

\raisebox{-\height}{

\includegraphics{03-methods_files/figure-pdf/fig-parameter-1.pdf}

}

}

\subcaption{\label{fig-parameter-1}100 draws}
\end{minipage}%
%
\begin{minipage}[t]{0.50\linewidth}

{\centering 

\raisebox{-\height}{

\includegraphics{03-methods_files/figure-pdf/fig-parameter-2.pdf}

}

}

\subcaption{\label{fig-parameter-2}600 draws}
\end{minipage}%

\caption{\label{fig-parameter}Sampled mode and destination choice
parameters for HBW trip purpose.}

\end{figure}

To determine if LHS is effective at a reasonable amount of iterations,
the cumulative mean and the cumulative standard deviation of the average
MCLS value for every zone (see Equation~\ref{eq-mcls} ) was calculated
for each additional draw for both sampling methods. MCLS is an impedance
term which is an important value for destination choice and region
routing. The average MCLS, \(x\), was used as a measure of outcome
possibilities to simplify a complex term as a single value to compare by
across all iterations. The cumulative mean is calculated as:
\begin{equation}\protect\hypertarget{eq-cmclsmean}{}{
\mu_i = \frac{x_1 + ... + x_i}{n}
}\label{eq-cmclsmean}\end{equation} and the cumulative standard
deviation is calculated as:
\begin{equation}\protect\hypertarget{eq-sdi}{}{
SD_i = \sqrt{\frac{\sum (x_i - \mu_i)^2 }{n-1}}.
}\label{eq-sdi}\end{equation} The cumulative mean shows how the average
MCLS stabilizes across each iteration, and the cumulative standard
deviation is used to show the 95\% confidence interval of that mean.
When the cumulative mean for the draws stabilizes, that shows that the
amount of generated parameters has captured the possible variance of the
results. This is shown for each of the three trip purposes in
Figure~\ref{fig-cm}.

\begin{figure}

\begin{minipage}[t]{0.50\linewidth}

{\centering 

\raisebox{-\height}{

\includegraphics{03-methods_files/figure-pdf/fig-cm-1.pdf}

}

}

\subcaption{\label{fig-cm-1}HBW}
\end{minipage}%
%
\begin{minipage}[t]{0.50\linewidth}

{\centering 

\raisebox{-\height}{

\includegraphics{03-methods_files/figure-pdf/fig-cm-2.pdf}

}

}

\subcaption{\label{fig-cm-2}HBO}
\end{minipage}%
\newline
\begin{minipage}[t]{0.50\linewidth}

{\centering 

\raisebox{-\height}{

\includegraphics{03-methods_files/figure-pdf/fig-cm-3.pdf}

}

}

\subcaption{\label{fig-cm-3}NHB}
\end{minipage}%

\caption{\label{fig-cm}Average mode choice logsum (impedance) cumulative
mean and 95\% confidence interval with 100 and 600 draws.}

\end{figure}

For all three trip purposes, both sampling methods had a stabilized mean
by 100 draws. The LHS methods standard deviation ribbon was generally
thinner than the MC method. From the narrowed cumulative standard
deviation, and that the parameter values are better normally distributed
when using LHS, that method of sampling was used for the assignment
analysis of the model. Since LHS captures the possible variance at a
small enough number of iterations, it can be used for large
transportation demand models. From these results it was decided to use
100 LHS samples parameters to evaluate uncertainty within each step of
the model. The next chapter includes the results of applying these
sampled parameters to the travel demand model.

\bookmarksetup{startatroot}

\hypertarget{sec-results}{%
\section{Sensitivity Analysis Results}\label{sec-results}}

Each of the 100 LHS parameter draws was applied to the RVTPO model,
generating mode choice utilities, destination choice utilities, and trip
matrices for each draw. The resulting uncertainty can then be quantified
using the outputs from the trip-based model. This section will first
look at the uncertainty of trips by mode, and how the mode split changes
when the parameters vary. Then uncertainty will be quantified using the
highway assigned trips, and how link volume changes across each draw.
The results will then be summarized.

\hypertarget{mode-choice-trips}{%
\subsection{Mode Choice Trips}\label{mode-choice-trips}}

Uncertainty can be evaluated by looking at how mode choices change. The
total number of trips by purpose are fixed, but the number of trips by
each mode changes as a result of mode choice, combined with the
availability of modes in the travel time skims. Table~\ref{tbl-mctrips}
lists the base trip amount by mode and purpose. It also lists the the
average number of trips across all 100 iterations, with the
corresponding standard deviation and coefficient of variation. For HBW
trips there are 103,320 auto trips. Across all 100 iterations there is a
mean value of 103,298 trips with a standard deviation of 527.07. This
results in a coefficient of variation of 0.0052 or 0.52\% variation in
the number of auto trips. The other modes of transportation are included
and similar patterns can be seen in HBO and NHB. The results listed in
the table show that the variation of the output trips - by mode and
purpose - are less than the input variation (as all \(c_v\)'s are
smaller than 0.10). This confirms previous research that the outcome
variance is less than or near the parameters variance (\emph{7},
\emph{9}). In all three purposes that were evaluated, the coefficient of
variation in auto trips are lower than transit or non-motorized trips,
meaning that there is greater confidence in the models accuracy to
generate auto trips. The input parameter variability has a smaller
effect on auto trips than on trips on the other modes.

\hypertarget{tbl-mctrips}{}
\begin{table}
\caption{\label{tbl-mctrips}Coefficient of Variation of Trips by Mode }\tabularnewline

\centering
\begin{tabular}{lrrrr}
\toprule
 & Base & Mean & SD & $c_v$\\
\midrule
\addlinespace[0.3em]
\multicolumn{5}{l}{\textbf{HBW}}\\
\hspace{1em}Auto & 103320 & 103298 & 537.07 & 0.0052\\
\hspace{1em}Non-Motorized & 1103 & 1105 & 50.38 & 0.0456\\
\hspace{1em}Transit & 13254 & 13274 & 566.01 & 0.0426\\
\addlinespace[0.3em]
\multicolumn{5}{l}{\textbf{HBO}}\\
\hspace{1em}Auto & 250489 & 250475 & 453.11 & 0.0018\\
\hspace{1em}Non-Motorized & 4310 & 4316 & 235.24 & 0.0545\\
\hspace{1em}Transit & 9276 & 9283 & 363.09 & 0.0391\\
\addlinespace[0.3em]
\multicolumn{5}{l}{\textbf{NHB}}\\
\hspace{1em}Auto & 60212 & 60209 & 78.28 & 0.0013\\
\hspace{1em}Non-Motorized & 736 & 737 & 35.77 & 0.0485\\
\hspace{1em}Transit & 1576 & 1579 & 74.89 & 0.0474\\
\bottomrule
\end{tabular}
\end{table}

The variation among mode choices can be visualized graphically using a
density of a scaled change in trips by mode.
Figure~\ref{fig-modechoicetrips} shows density plots for HBW trips by
mode for 12 zones -- the zones are divided into three volume categories:
low is less than 200 trips per zone, mid is 200 to 700 trips per zone,
and top is greater than 700 trips per zone -- and four zones are
randomly selected from each volume category. Zones that do not have any
transit accessibility have been excluded. Those zones have very high
density in auto trips as with the ability to choose transit was removed,
the choice to choose auto was more certain. The zones included in
Figure~\ref{fig-modechoicetrips} all have greater certainty in auto
trips, as the change in trips across all 100 iterations is relatively
small. This reinforces the previous claim that the model has more
confidence in auto trips than the other modes. It is also important to
note that the modes are correlated to each other. In zones with a
greater confidence in one mode, the other modes are more confident as
well. Since the number of trips by origin zone are held constant, when
there are an increase in trips on one mode there must be a decrease in
trips on one or both of the other modes. Also, the distribution of
non-motorized trips is similar for every zone suggesting that generally,
the most variable mode is non-motorized trips which you can see in the
spread of the graphic. This is also verified using
Table~\ref{tbl-mctrips} as the \(c_v\) is largest for the non-motorized
mode across all three purposes.

\begin{sidewaysfigure}

{\centering \includegraphics{04-results_files/figure-pdf/fig-modechoicetrips-1.pdf}

}

\caption{\label{fig-modechoicetrips}Trip density for coefficient of
variation by mode for HBW trips.}

\end{sidewaysfigure}

\hypertarget{link-volume}{%
\subsection{Link Volume}\label{link-volume}}

Highway volumes are the most commonly used output of a travel model.
Uncertainty can additionally be evaluated by looking at how assigned
link volume varies across iterations. Figure~\ref{fig-networksd}
displays variation in forecast link volume spatially. This shows that
the links with the highest standard deviation in forecast volume are
high-volume roads including freeways and principal arterials where the
majority of traffic is internal to the study region. Although these
links have the largest standard deviation, when compared to the total
volume of the road, the variation is in reality very small. A standard
deviation of 400 vehicles on a road with 40,000 total vehicles
corresponds to a small variation (1\%).

\begin{sidewaysfigure}

{\centering \includegraphics{04-results_files/figure-pdf/fig-networksd-1.pdf}

}

\caption{\label{fig-networksd}Standard deviation in daily forecast
volume.}

\end{sidewaysfigure}

The highway assignment results can be grouped by facility type to show
how the coefficient of variation compares to link volume.
Figure~\ref{fig-totalvolume} shows the coefficient of variation for the
daily volume assigned to each network link, across the 100 draws,
plotted against the mean forecast link volume for each link. The values
are the volume for 100 randomly sampled links for each facility type.
The plots shows that for the high-volume roads such as major arterials
and freeways, the coefficient of variation converges to approximately
0.01, or about 1\% of the road's total forecast volume. For lower-volume
links, the coefficient of variation is more widely distributed, with
some local roads and small collectors having considerably higher values.
Some links in the model show no variation at all; these are presumably
links near the edges of the model region where the only traffic is to
and from external zones, trips which were held constant in this
framework.

\begin{figure}

{\centering \includegraphics[width=0.8\textwidth,height=\textheight]{04-results_files/figure-pdf/fig-totalvolume-1.pdf}

}

\caption{\label{fig-totalvolume}Coefficient of variation in daily link
volume by facility type for a random sample of highway links.}

\end{figure}

Variation among a link can also be visualized with a density plot of the
total volume across all iterations, as shown in
Figure~\ref{fig-densityplots}. In this plot, the density of forecast
volumes in three randomly selected links in each of the freeway,
collector, and arterial functional types are plotted alongside the
baseline forecast and the Average Annual Weekday Daily Traffic (AAWDT)
measured by the Virginia Department of Transportation, and to which the
model estimates were calibrated. In all cases, the error or uncertainty
in the forecast is considerably narrower than the error inherent in the
model construction, as evidenced by the fact that the AAWDT target value
is well outside the bell curve created by the statistically varied
simulation forecasts.

As expected from using the base parameter values as the mean of the LHS
parameter sampling, the base results are at or near the median of the
statistical density for each link's volume. But it is notable that the
estimated volumes are not perfectly, normally distributed as might be
naively expected. In this case, the combined effects of the mode and
destination choice parameter sampling appear to be constrained by the
geographic specificity of the RVTPO model network: even when the demand
for trips changes between zone pairs, the realities of the highway
capacity, volume-delay, and static user equilibrium procedures may be
limiting the possibilities for forecast highway volumes.

\begin{figure}

{\centering \includegraphics[width=0.8\textwidth,height=\textheight]{04-results_files/figure-pdf/fig-densityplots-1.pdf}

}

\caption{\label{fig-densityplots}Density plot of forecast volume on
selected links, with default parameter results marked in red, and AAWDT
values in green.}

\end{figure}

\bookmarksetup{startatroot}

\hypertarget{sec-conclusions}{%
\section{Conclusions}\label{sec-conclusions}}

The uncertainty and variance present in the model results are overall
smaller than the variability introduced to the model through the
parameters. Auto trips have less variance than transit or non-motorized
trips, regardless of the number of trips in a given zone. When there is
only two modes available auto trips have even lower variance. Higher
volume roads have greater certainty than lower volume ones. Also,
uncertainty that exists within the first three steps of the model
appears to be corrected by the limitations of the highway network
assignment. Overall there is a variance of approximately 1\% in the
model outputs, when a 10 percent variance was used for the parameter
estimates.

In general, this research has shown that statistical parameter
uncertainty does not appear to be a significant factor in forecasting
traffic volumes using transportation demand models. The result
uncertainty is generally equally to or smaller than the input parameter
variance. The uncertainty in parameter inputs appears to lead to
variation in highway volumes that is lower than the error between the
model forecast and the highway counts. Any variation in mode and
destination choice probabilities appears to be constrained by the
limitations of the highway network assignment.

There are several limitations that must be mentioned in this research,
however. First, I did not attempt to address the statistical uncertainty
in trip rate estimates; these may play a substantially larger role than
destination and mode choice parameters, given that lower trip rates may
lead to lower traffic volumes globally, which could not be ``corrected''
by the static user equilibrium assignment. And the assignment itself
might have been aided in this case by the RVTPO's relatively simplistic
highway network: if only two major highways exist the number of
realistic lowest cost path choices between regions is somewhat
constrained. It may be that in a larger network with more path
redundancies, the assignment may not have been as helpful in
constraining the forecasted volumes.

In this research I had only the estimates of the statistical
coefficients, and therefore had to assume a coefficient of variation to
derive variation in the sampling procedure. It would be better if model
user and development documentation more regularly provided estimates of
the standard errors of model parameters. Even better would be
variance-covariance matrices for the estimated models, enabling
researchers to ensure that covariance relationships between sampled
parameters are maintained.

Notwithstanding these limitations, statistical parameter variance is not
likely the largest source of uncertainty in travel forecasting. There
are likely more important factors at play that planners and government
agencies should address. Research on all sources of uncertainty is
somewhat limited, but in many ways has been hampered by the burdensome
computational requirements of many modern travel models (\emph{2}). This
research methodology benefited from a lightweight travel model that
could be repeatedly re-run with dozens of resampled choice parameters.
Better understanding the other sources of uncertainty -- model
specification and input accuracy -- might also benefit from lightweight
models constructed for transparency and flexibility rather than heavily
constrained models emphasizing precise spatial detail and strict
behavioral constraints. This might allow forecasts to be made with an
ensemble approach (\emph{20}), identifying preferred policies as the
consensus of multiple plausible model specifications.

\bookmarksetup{startatroot}

\hypertarget{references}{%
\section*{References}\label{references}}
\addcontentsline{toc}{section}{References}

\markboth{References}{References}

\hypertarget{refs}{}
\begin{CSLReferences}{0}{0}
\leavevmode\vadjust pre{\hypertarget{ref-flyvbjerg2005}{}}%
\CSLLeftMargin{1. }%
\CSLRightInline{Flyvbjerg, B., M. K. Skamris Holm, and S. L. Buhl. How
({In})accurate {Are Demand Forecasts} in {Public Works Projects}?: {The
Case} of {Transportation}. \emph{Journal of the American Planning
Association}, Vol. 71, No. 2, 2005, pp. 131--146.
\url{https://doi.org/10.1080/01944360508976688}.}

\leavevmode\vadjust pre{\hypertarget{ref-voulgaris2019}{}}%
\CSLLeftMargin{2. }%
\CSLRightInline{Voulgaris, C. T. Crystal {Balls} and {Black Boxes}:
{What Makes} a {Good Forecast}? \emph{Journal of Planning Literature},
Vol. 34, No. 3, 2019, pp. 286--299.
\url{https://doi.org/10.1177/0885412219838495}.}

\leavevmode\vadjust pre{\hypertarget{ref-rasouli2012}{}}%
\CSLLeftMargin{3. }%
\CSLRightInline{Rasouli, S., and H. Timmermans. Uncertainty in Travel
Demand Forecasting Models: Literature Review and Research Agenda.
\emph{Transportation Letters}, Vol. 4, No. 1, 2012, pp. 55--73.
\url{https://doi.org/10.3328/TL.2012.04.01.55-73}.}

\leavevmode\vadjust pre{\hypertarget{ref-hoque2021}{}}%
\CSLLeftMargin{4. }%
\CSLRightInline{Hoque, J. M., G. D. Erhardt, D. Schmitt, M. Chen, and M.
Wachs. Estimating the Uncertainty of Traffic Forecasts from Their
Historical Accuracy. \emph{Transportation Research Part A: Policy and
Practice}, Vol. 147, 2021, pp. 339--349.
\url{https://doi.org/10.1016/j.tra.2021.03.015}.}

\leavevmode\vadjust pre{\hypertarget{ref-koppelman2006}{}}%
\CSLLeftMargin{5. }%
\CSLRightInline{Koppelman, F. S., and C. Bhat. \emph{A {Self Instructing
Course} in {Mode Choice Modeling}: {Multinomial} and {Nested Logit
Models}}. {Federal Transit Administration}, 2006.}

\leavevmode\vadjust pre{\hypertarget{ref-nationalacademiesofsciencesengineeringandmedicine.2012}{}}%
\CSLLeftMargin{6. }%
\CSLRightInline{National Academies of Sciences, Engineering, and
Medicine. \emph{\href{https://doi.org/10.17226/14665}{Travel {Demand
Forecasting}: {Parameters} and {Techniques}}}. Publication NCHRP 716.
{National Academies Press}, {Washington, D.C.}, 2012.}

\leavevmode\vadjust pre{\hypertarget{ref-zhao2002}{}}%
\CSLLeftMargin{7. }%
\CSLRightInline{Zhao, Y., and K. M. Kockelman. The Propagation of
Uncertainty Through Travel Demand Models: {An} Exploratory Analysis.
\emph{The Annals of Regional Science}, Vol. 36, No. 1, 2002, pp.
145--163. \url{https://doi.org/10.1007/s001680200072}.}

\leavevmode\vadjust pre{\hypertarget{ref-rodier2002uncertain}{}}%
\CSLLeftMargin{8. }%
\CSLRightInline{Rodier, C. J., and R. A. Johnston. Uncertain
Socioeconomic Projections Used in Travel Demand and Emissions Models:
Could Plausible Errors Result in Air Quality Nonconformity?
\emph{Transportation Research Part A: Policy and Practice}, Vol. 36, No.
7, 2002, pp. 613--631.}

\leavevmode\vadjust pre{\hypertarget{ref-clay2005univariate}{}}%
\CSLLeftMargin{9. }%
\CSLRightInline{Clay, M. J., and R. A. Johnston. Univariate Uncertainty
Analysis of an Integrated Land Use and Transportation Model: {MEPLAN}.
\emph{Transportation Planning and Technology}, Vol. 28, No. 3, 2005, pp.
149--165.}

\leavevmode\vadjust pre{\hypertarget{ref-armoogum2009}{}}%
\CSLLeftMargin{10. }%
\CSLRightInline{Armoogum, J., J.-L. Madre, and Y. Bussiere. Measuring
Uncertainty in Long-Term Travel Demand Forecasting from Demographic
Modelling: {Case} Study of the {Paris} and {Montreal} Metropolitan
Areas. \emph{IATSS research}, Vol. 33, No. 2, 2009, pp. 9--20.}

\leavevmode\vadjust pre{\hypertarget{ref-duthie2010highway}{}}%
\CSLLeftMargin{11. }%
\CSLRightInline{Duthie, J., A. Voruganti, K. Kockelman, and S. T.
Waller. Highway Improvement Project Rankings Due to Uncertain Model
Inputs: {Application} of Traditional Transportation and Land Use Models.
\emph{Journal of Urban Planning and Development}, Vol. 136, No. 4, 2010,
pp. 294--302.}

\leavevmode\vadjust pre{\hypertarget{ref-welde2011planners}{}}%
\CSLLeftMargin{12. }%
\CSLRightInline{Welde, M., and J. Odeck. Do Planners Get It Right? {The}
Accuracy of Travel Demand Forecasting in {Norway}. \emph{European
Journal of Transport and Infrastructure Research}, Vol. 11, No. 1,
2011.}

\leavevmode\vadjust pre{\hypertarget{ref-yang2013}{}}%
\CSLLeftMargin{13. }%
\CSLRightInline{Yang, C., A. Chen, X. Xu, and S. C. Wong.
Sensitivity-Based Uncertainty Analysis of a Combined Travel Demand
Model. \emph{Transportation Research Part B: Methodological}, Vol. 57,
2013, pp. 225--244. \url{https://doi.org/10.1016/j.trb.2013.07.006}.}

\leavevmode\vadjust pre{\hypertarget{ref-manzo2015}{}}%
\CSLLeftMargin{14. }%
\CSLRightInline{Manzo, S., O. A. Nielsen, and C. G. Prato. How
Uncertainty in Input and Parameters Influences Transport Model: Output
{A} Four-Stage Model Case-Study. \emph{Transport Policy}, Vol. 38, 2015,
pp. 64--72.}

\leavevmode\vadjust pre{\hypertarget{ref-petrik2016measuring}{}}%
\CSLLeftMargin{15. }%
\CSLRightInline{Petrik, O., F. Moura, and J. de A. e Silva. Measuring
Uncertainty in Discrete Choice Travel Demand Forecasting Models.
\emph{Transportation Planning and Technology}, Vol. 39, No. 2, 2016, pp.
218--237.}

\leavevmode\vadjust pre{\hypertarget{ref-petrik2020uncertainty}{}}%
\CSLLeftMargin{16. }%
\CSLRightInline{Petrik, O., M. Adnan, K. Basak, and M. Ben-Akiva.
Uncertainty Analysis of an Activity-Based Microsimulation Model for
{Singapore}. \emph{Future Generation Computer Systems}, Vol. 110, 2020,
pp. 350--363.}

\leavevmode\vadjust pre{\hypertarget{ref-aep50_2023}{}}%
\CSLLeftMargin{17. }%
\CSLRightInline{AEP50 Committee on Transportation Demand Forecasting.
Uncertainty. https://www.trbtravelforecasting.org/uncertainty, 2023.}

\leavevmode\vadjust pre{\hypertarget{ref-ben-akiva1985}{}}%
\CSLLeftMargin{18. }%
\CSLRightInline{Ben-Akiva, M., and S. R. Lerman.
\emph{\href{https://www.jstor.org/stable/1391567}{Discrete {Choice
Analysis}: {Theory} and {Applications} to {Travel Demand}}}. {MIT
Press}, 1985.}

\leavevmode\vadjust pre{\hypertarget{ref-helton2003a}{}}%
\CSLLeftMargin{19. }%
\CSLRightInline{Helton, J. C., and F. J. Davis. Latin Hypercube Sampling
and the Propagation of Uncertainty in Analyses of Complex Systems.
\emph{Reliability Engineering \& System Safety}, Vol. 81, No. 1, 2003,
pp. 23--69. \url{https://doi.org/10.1016/S0951-8320(03)00058-9}.}

\leavevmode\vadjust pre{\hypertarget{ref-wu2021}{}}%
\CSLLeftMargin{20. }%
\CSLRightInline{Wu, H., and D. Levinson. The Ensemble Approach to
Forecasting: {A} Review and Synthesis. \emph{Transportation Research
Part C: Emerging Technologies}, Vol. 132, 2021, p. 103357.
\url{https://doi.org/10.1016/j.trc.2021.103357}.}

\end{CSLReferences}



\end{document}
