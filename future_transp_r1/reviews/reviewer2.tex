% An example for the ar2rc document class.
% Copyright (C) 2017 Martin Schroen
% Modifications Copyright (C) 2020 Kaishuo Zhang
%
% This program is free software: you can redistribute it and/or modify
% it under the terms of the GNU General Public License as published by
% the Free Software Foundation, either version 3 of the License, or
% (at your option) any later version.
%
% This program is distributed in the hope that it will be useful,
% but WITHOUT ANY WARRANTY; without even the implied warranty of
% MERCHANTABILITY or FITNESS FOR A PARTICULAR PURPOSE.  See the
% GNU General Public License for more details.
%
% You should have received a copy of the GNU General Public License
% along with this program.  If not, see <http://www.gnu.org/licenses/>.

\documentclass{ar2rc}

\title{Evaluating the Impacts of Parameter Uncertainty in a Practical Transportation Demand Model}
\author{G. Macfarlane}
\journal{Future Transportation}
\doi{futuretransp-3339770}

\begin{document}

\maketitle

We are grateful for the reviewer's comments on this manuscript. We address each point in turn, with 
\DIFaddbegin \DIFadd{text added to the manuscript in blue}\DIFaddend  and 
\DIFdelbegin \DIFdel{text deleted from the manuscript in red}\DIFdelend.
We address each 

\section{Reviewer \#2}

\subsection{Introduction}
\RC This manuscript presents a detailed investigation into the impacts of parameter uncertainty
in a practical, trip-based transportation demand model using the Roanoke Valley
Transportation Planning Organization (RVTPO) model as a case study. By employing Latin
Hypercube Sampling (LHS) to explore parameter variations and their impact on traffic
forecasts, the research addresses a critical issue in transportation modelling: the
robustness of parameter uncertainty in realistic scenarios. In sum, the study provides a
clear methodology and robust results, showing that parameter uncertainty has a minimal
effect on traffic volumes in this specific model. The findings contribute to the understanding
of uncertainties in transportation demand forecasting and their implications for decision-
making in infrastructure planning.
From the point of view of relevance and originality, I think the paper tackles a timely topic,
as accurate transportation demand forecasts are essential for policy and infrastructure
planning. It particularly focuses on parameter uncertainty, a less explored but highly
impactful area within transportation modelling literature.

\AR We are grateful for the reviewer's positive comments on the originality and timeliness 
of this research.

\subsection{Literature Review}
\RC While the literature review covers a broad range of studies, it could be more focused. For
instance, the review mentions several papers that primarily address input data or model
form uncertainties, which are less relevant to the paper’s primary focus on parameter
uncertainty. A deeper dive into recent studies specifically addressing parameter
uncertainty in trip-based models or activity-based frameworks would strengthen the
contextual relevance.

\AR We agree with this general point, but feel it is important to highlight that research 
has largely been focused on things other than parameter uncertainty, which we identify in 
the paper's Introduction as the only element of modeling addressed by classical statistics.


\subsection{Methodology}
\RC I also consider that the methodology employed by the authors to assess parameter
uncertainty is comprehensive and adequate. Their use of Latin Hypercube Sampling (LHS)
to construct hundreds of combinations of parameters across a plausible parameter space
allows for a more efficient sampling of the parameter space compared to simple random
sampling, ensuring that the entire range of possible values is explored. This is a notable
strength, as well as their evaluation across multiple trip purposes and the inclusion of high-
volume and low-volume network links, to provide a comprehensive sensitivity analysis. The
authors also introduced substantial changes to implied travel impedances and modal
utilities based on the sampled parameter combinations, allowing them to observe the
effects on traffic volume forecasts.

\AR We are glad that the reviewer found this to be a strength of the paper, and we agree that 
it leads to the strongest contribution of the paper.


\subsection{Results}
\RC In my opinion the results are well-structured, with clear presentation in tables and graphs.
The conclusion—that parameter uncertainty contributes minimally to forecast variation
compared to network and model constraints—is supported by quantitative evidence. In
fact, the study's findings suggest that efforts to address uncertainties in travel forecasting
may be better spent on improving model specifications and input data accuracy rather than
focusing solely on parameter uncertainty.

\AR We are happy that the reviewer found our results persuasive in light of the evidence we submit.

\subsection{Minor Comments}

\subsubsection{Practical Impacts}
\RC The abstract effectively summarizes the study, but including a line on the practical
implications of the findings would enhance its appeal.

\AR We welcome this advice and have revised the abstract,

\begin{quote}
     Using Latin hypercube sampling to
construct several hundred combinations of parameters across the
plausible parameter space, we introduce substantial changes to implied
travel impedances and modal utilities\DIFaddbegin \DIFadd{, on the order of 10 percent variation}\DIFaddend .
However, the aggregate effects of of these changes on forecasted traffic volumes
is small, with a \DIFdelbegin \DIFdel{variance
}\DIFdelend \DIFaddbegin \DIFadd{variation }\DIFaddend of approximately 1 percent on high-volume facilities.
It is likely that in this example --- and perhaps in others --- the \DIFdelbegin \DIFdel{static }\DIFdelend network
assignment places constraints on the possible volume solutions and limits the
practical impacts of parameter uncertainty. \DIFaddbegin \DIFadd{Nevertheless, parameter uncertainty
may not be the largest contributor to error in practical travel forecasts. }\DIFaddend Further
research should examine the robustness of this finding to other less constrained
networks and to activity-based travel model frameworks.
\end{quote}

\subsubsection{Coefficient of variation}
\RC The paper assumes a coefficient of variation (CV) of 0.10 for parameter uncertainty.
While the authors justify this with a rational range for value of time, the choice could be
further validated with sensitivity tests or references to empirical studies. In Table 3, the
coefficient of variation for non-motorized and transit trips is substantially higher than
for auto trips. This observation could be highlighted and discussed in the results
section to emphasize differences in confidence across modes.

\AR We unfortunately do not have the resources to conduct an additional
sensitivity analysis, which we agree would be valuable. We have, however, revised this paragraph
to include an additional external reference in the literature,

\begin{quote}
    With the trip-based model described above, MC and LHS methods were used
    to develop alternative parameter sets to evaluate uncertainty. To
    identify a standard deviation for each parameter, \DIFaddbegin \DIFadd{we asserted }\DIFaddend a coefficient of
    variation \DIFdelbegin \DIFdel{was used. A set coefficient of variation of 0.10 was used for
    }\DIFdelend \DIFaddbegin \DIFadd{\(c_v = 0.10 \) }\DIFaddend the four mode choice coefficients and the destination
    choice parameters\DIFdelbegin \DIFdel{.
    The }\DIFdelend \DIFaddbegin \DIFadd{; the }\DIFaddend mode choice constants \DIFdelbegin \DIFdel{were kept the same }\DIFdelend \DIFaddbegin \DIFadd{remained fixed }\DIFaddend across all iterations.
    Literature had identified a coefficient of variation of 0.30 \DIFdelbegin \DIFdel{, }\DIFdelend \DIFaddbegin \DIFadd{(Zhao and Kockelman, 2002), }\DIFaddend but for
    this analysis that caused an unrealistic value of time, and thus it was
    changed to be 0.10\DIFdelbegin \DIFdel{(Zhao and Kockelman, 2002)}\DIFdelend . Value of time is a ratio
    in units of money per time that should be compared to the regional wage
    rate \DIFdelbegin \DIFdel{. Using a }\DIFdelend \DIFaddbegin \DIFadd{and was generally on the order of 10 dollars per hour in the early 2010's (Abrantes and
    Wardman, 2011), when the RVTPO choice coefficients were
    developed.
     A }\DIFaddend \(c_v\) of 0.30 \DIFdelbegin \DIFdel{the }\DIFdelend \DIFaddbegin \DIFadd{implied a }\DIFaddend value of time
    \DIFdelbegin \DIFdel{range was }\DIFdelend \DIFaddbegin \DIFadd{extending }\DIFaddend from \(\$2\)
    to \(\$32\) \DIFdelbegin \DIFdel{/hr, whereas using }\DIFdelend \DIFaddbegin \DIFadd{per hour, whereas }\DIFaddend a \(c_v\) of 0.10 \DIFdelbegin \DIFdel{the range was }\DIFdelend \DIFaddbegin \DIFadd{implied values between }\DIFaddend \(\$6\)
    to \(\$14\) /hr\DIFdelbegin \DIFdel{. The latter seemed more rational because it is related
    to wage rates and thus a \(c_v\) of 0.10 was used for our analysis}\DIFdelend \DIFaddbegin \DIFadd{, which we assess as more reasonable for this context}\DIFaddend .
    The standard deviation \DIFdelbegin \DIFdel{was }\DIFdelend \DIFaddbegin \DIFadd{for sampling the parameters was therefore }\DIFaddend equal to 0.10
    multiplied by the mean, where the mean values in this situation are the base
    scenario parameters (as identified in Table 2).
\end{quote}

\AR To be specific, the Abrantes and Wardman reference contains a meta-analysis of value of time studies
conducted in the United Kingdom. Basically all mean values are concentrated 
around 10 UK pence per minute, which is effectively equal to dollars per hour. Considering 
\begin{equation}
  1 \frac{\mathrm{pence}}{\mathrm{minute}} \times \frac{60\ \mathrm{minutes}}{\mathrm{hr}} 
    \times \frac{\mathrm{pound\ sterling}}{100 \mathrm{\ pence}} \times \frac{1 \mathrm{\ USD}}{0.65 \mathrm{\ pound\ sterling}}
\end{equation}
where the equivalence between pounds and dollars is based on purchasing power parity (data from OECD).

\subsubsection{Limitations}
\RC The authors note that the RVTPO model's size and constraints may limit the
generalizability of the findings. However, further discussion on how these limitations
impact other models, especially in large urban or multimodal contexts, would add
depth.

\RC [moved up from below] The findings are based on a specific trip-based travel demand model, which
may not be applicable to all contexts or regions. The results might vary
significantly in different geographic areas or under different modelling
frameworks, such as activity-based models.

\AR We agree that an additional limitation is warranted, and we have added this to the conclusions,

\begin{quote}
    \DIFaddbegin \DIFadd{In general, these findings are based on a specific trip-based travel
    demand model, which may not be applicable to all contexts or regions. The
    results might vary significantly in different geographic areas or under
    different modeling frameworks, such as activity-based models. Attempting this 
    research again with a variety of models and geographic regions would be a valuable research
    priority.
    }\DIFaddend 
\end{quote}

\subsubsection{Future research}
\RC The conclusion section could expand on how these findings might influence future
research priorities or practical applications in urban transportation planning.

\AR We have revised the conclusions section to address future research in this area,

\begin{quote}
    In this research we had only the estimates of the statistical
coefficients, and therefore had to assume a coefficient of variation to
derive variation in the sampling procedure. It would be better if model
user and development documentation more regularly provided estimates of
the standard errors of model parameters. \DIFdelbegin \DIFdel{Even better }\DIFdelend \DIFaddbegin \DIFadd{The ideal }\DIFaddend would be
variance-covariance matrices for the estimated models, enabling
researchers to ensure that covariance relationships between sampled
parameters are maintained. \DIFaddbegin \DIFadd{Future research might reconsider the present 
expermiment but allowing for correlation between parameter values.
}\DIFaddend 

\end{quote}

\subsubsection{Equation and figure discussion}
\RC Certain equations, such as those describing mode and destination choice, could
benefit from additional explanation or context to ensure accessibility to readers
unfamiliar with specific modelling practices. Figures, while generally effective, could
include more explicit labels or annotations for non-specialist readers. Figure 3 on trip
density by mode is insightful but could be paired with more textual analysis to discuss
patterns by trip purpose or link volume.

\AR We attempted to re-word the description of the mode and destination choice
sections to render them more generally understood,

\begin{quote}
    The utility equations for the mode choice model are as follows: \[
\begin{aligned}
U_{auto} &= \beta_{ivtt} * X_{auto} + \beta_{tc} * \beta_{ac} * X_{dist}\\
U_{nmot} &= k_{nmot} + 20 * \beta_{wd}*X_{nmot}\\
U_{trn} &= k_{trn} + \beta_{ivtt} * X_{trn}
\end{aligned}
\] \DIFdelbegin \DIFdel{These utilities are used to calculate the MCLS by:
}\begin{displaymath}\DIFdel{\phantomsection%DIFDELCMD < \label{eq-mcls}%%%
{
MCLS_{ij} = \ln\left(\sum_{k \in K} e^{U_{ijk}}\right).
}}\end{displaymath}%DIFAUXCMD
%DIFDELCMD <  %%%
\DIFdelend If the distance was greater than 2 miles, non-motorized
travel was excluded \DIFdelbegin \DIFdel{.
}%DIFDELCMD < 

%DIFDELCMD < %%%
\DIFdelend \DIFaddbegin \DIFadd{as an option. In general, modes with longer times receive lower probabilities and therefore lower proportions of trips.
These utilities are used to calculate the MCLS by:
}\begin{equation}\DIFadd{\phantomsection\label{eq-mcls}{
MCLS_{ij} = \ln\left(\sum_{k \in K} e^{U_{ijk}}\right).
}}\end{equation}  \DIFaddend This logsum value is then used as the primary impedance for a
destination choice model (Ben-Akiva and Lerman, 1985). 
\DIFdelbegin \DIFdel{Destination
choice estimates travel patterns based on mode choice,
trip generators
(workers and households}\DIFdelend \DIFaddbegin 

\DIFadd{The destination choice model estimates the numbers of trips between origin and
destination pairs using the size of the destination (number of attractions),
accessibility by multiple modes (in the MCLS}\DIFaddend ),
\end{quote}


\subsubsection{Other sources of uncertainty}
\RC The study primarily addresses uncertainty related to mode and destination
choice parameters, without exploring the statistical uncertainty in trip
production estimates. This omission could mean that other significant sources
of uncertainty affecting traffic volumes are not adequately considered.

\AR This is correct, and has been noted as a limitation.

\RC [moved up from below] Finally, the paper does not delve deeply into other sources of uncertainty, such
as input data inaccuracies or model specification errors, which could also
significantly impact forecasting accuracy. A more comprehensive analysis of
these factors could provide a fuller understanding of the uncertainties involved
in transportation demand modelling.

\AR Again, we agree with this statement and have acknowledged it in the conclusion discussion. 

\subsubsection{Static assignment constraint}
\RC The research suggests that the static network assignment may constrain the
possible volume solutions, potentially limiting the practical impacts of
parameter uncertainty. This raises questions about the generalizability of the
findings to more dynamic or less constrained network scenarios.

\AR While we did say this, the real constraint comes not from the particular static
user equilibrium assignment but from the highway capacities themselves: there are 
only so many routes one can use to drive through a city, regardless of where the trips
are destined. We have expanded this in the conclusions,

\begin{quote}
    given that
    lower trip rates may lead to lower traffic volumes globally, which could
    not be ``corrected'' by the \DIFdelbegin \DIFdel{static user equilibrium assignment. }\DIFdelend \DIFaddbegin \DIFadd{traffic assignment. Second, a different
    methodology of sampling might have produced a different result at the extremes than
    the results of LHS.
    }\DIFaddend Additionally, the relatively sparse network of the RVTPO model region
    --- lacking parallel high-capacity highway facilities --- may have meant
    that the \DIFdelbegin \DIFdel{static network assignment }\DIFdelend \DIFaddbegin \DIFadd{network assignment process }\DIFaddend would converge to a similar solution
    point regardless of modest changes to the trip matrix. \DIFaddbegin \DIFadd{If the number of paths 
    between nodes is limited and constrained by highway capacity, there are only so many 
    solutions to any highway assignment process. }\DIFaddend It may be that in
    a larger network with more path redundancies \DIFdelbegin \DIFdel{, }\DIFdelend \DIFaddbegin \DIFadd{or more alternative transit services
    }\DIFaddend the assignment may not have been as helpful in constraining the forecast
    volumes.
\end{quote}




\end{document}
