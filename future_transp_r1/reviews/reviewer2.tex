% An example for the ar2rc document class.
% Copyright (C) 2017 Martin Schroen
% Modifications Copyright (C) 2020 Kaishuo Zhang
%
% This program is free software: you can redistribute it and/or modify
% it under the terms of the GNU General Public License as published by
% the Free Software Foundation, either version 3 of the License, or
% (at your option) any later version.
%
% This program is distributed in the hope that it will be useful,
% but WITHOUT ANY WARRANTY; without even the implied warranty of
% MERCHANTABILITY or FITNESS FOR A PARTICULAR PURPOSE.  See the
% GNU General Public License for more details.
%
% You should have received a copy of the GNU General Public License
% along with this program.  If not, see <http://www.gnu.org/licenses/>.

\documentclass{ar2rc}

\title{Evaluating the Impacts of Parameter Uncertainty in a Practical Transportation Demand Model}
\author{G. Macfarlane}
\journal{Future Transportation}
\doi{futuretransp-3339770}

\begin{document}

\maketitle

We are grateful for the reviewer's comments on this manuscript. We address each point in turn, with 
\DIFaddbegin \DIFadd{text added to the manuscript in blue}\DIFaddend  and 
\DIFdelbegin \DIFdel{text deleted from the manuscript in red}\DIFdelend.


\section{Reviewer \#2}

\subsection{Introduction}
\RC This manuscript presents a detailed investigation into the impacts of parameter uncertainty
in a practical, trip-based transportation demand model using the Roanoke Valley
Transportation Planning Organization (RVTPO) model as a case study. By employing Latin
Hypercube Sampling (LHS) to explore parameter variations and their impact on traffic
forecasts, the research addresses a critical issue in transportation modelling: the
robustness of parameter uncertainty in realistic scenarios. In sum, the study provides a
clear methodology and robust results, showing that parameter uncertainty has a minimal
effect on traffic volumes in this specific model. The findings contribute to the understanding
of uncertainties in transportation demand forecasting and their implications for decision-
making in infrastructure planning.
From the point of view of relevance and originality, I think the paper tackles a timely topic,
as accurate transportation demand forecasts are essential for policy and infrastructure
planning. It particularly focuses on parameter uncertainty, a less explored but highly
impactful area within transportation modelling literature.

\AR We are grateful for the reviewer's positive comments on the originality and timeliness 
of this research.

\subsection{Literature Review}
\RC While the literature review covers a broad range of studies, it could be more focused. For
instance, the review mentions several papers that primarily address input data or model
form uncertainties, which are less relevant to the paper’s primary focus on parameter
uncertainty. A deeper dive into recent studies specifically addressing parameter
uncertainty in trip-based models or activity-based frameworks would strengthen the
contextual relevance.

\AR We agree with this general point, but feel it is important to highlight that research 
has largely been focused on things other than parameter uncertainty, which we identify in 
the paper's Introduction as the only element of modeling addressed by classical statistics.

This comment did, however, cause us to investigate some interesting modern research in the 
application of Bayesian statistics to model development and calibration. We have added a brief
discussion in the literature review,

\begin{quote}

\end{quote}

\subsection{Methodology}
\RC I also consider that the methodology employed by the authors to assess parameter
uncertainty is comprehensive and adequate. Their use of Latin Hypercube Sampling (LHS)
to construct hundreds of combinations of parameters across a plausible parameter space
allows for a more efficient sampling of the parameter space compared to simple random
sampling, ensuring that the entire range of possible values is explored. This is a notable
strength, as well as their evaluation across multiple trip purposes and the inclusion of high-
volume and low-volume network links, to provide a comprehensive sensitivity analysis. The
authors also introduced substantial changes to implied travel impedances and modal
utilities based on the sampled parameter combinations, allowing them to observe the
effects on traffic volume forecasts.

\AR We are glad that the reviewer found this to be a strength of the paper, and we agree that 
it leads to the strongest contribution of the paper.


\subsection{Results}
\RC In my opinion the results are well-structured, with clear presentation in tables and graphs.
The conclusion—that parameter uncertainty contributes minimally to forecast variation
compared to network and model constraints—is supported by quantitative evidence. In
fact, the study's findings suggest that efforts to address uncertainties in travel forecasting
may be better spent on improving model specifications and input data accuracy rather than
focusing solely on parameter uncertainty.

\AR We are happy that the reviewer found our results persuasive in light of the evidence we submit.

\subsection{Minor Comments}

\subsubsection{Practical Impacts}
\RC The abstract effectively summarizes the study, but including a line on the practical
implications of the findings would enhance its appeal.

\subsubsection{Coefficient of variation}
\RC The paper assumes a coefficient of variation (CV) of 0.10 for parameter uncertainty.
While the authors justify this with a rational range for value of time, the choice could be
further validated with sensitivity tests or references to empirical studies. In Table 3, the
coefficient of variation for non-motorized and transit trips is substantially higher than
for auto trips. This observation could be highlighted and discussed in the results
section to emphasize differences in confidence across modes.

\subsubsection{Limitations}
\RC The authors note that the RVTPO model's size and constraints may limit the
generalizability of the findings. However, further discussion on how these limitations
impact other models, especially in large urban or multimodal contexts, would add
depth.

\subsubsection{Future research}
\RC The conclusion section could expand on how these findings might influence future
research priorities or practical applications in urban transportation planning.

\subsubsection{Equation and figure discussion}
\RC Certain equations, such as those describing mode and destination choice, could
benefit from additional explanation or context to ensure accessibility to readers
unfamiliar with specific modelling practices. Figures, while generally effective, could
include more explicit labels or annotations for non-specialist readers. Figure 3 on trip
density by mode is insightful but could be paired with more textual analysis to discuss
patterns by trip purpose or link volume.


\subsubsection{Other sources of uncertainty}
\RC The study primarily addresses uncertainty related to mode and destination
choice parameters, without exploring the statistical uncertainty in trip
production estimates. This omission could mean that other significant sources
of uncertainty affecting traffic volumes are not adequately considered.

\RC Finally, the paper does not delve deeply into other sources of uncertainty, such
as input data inaccuracies or model specification errors, which could also
significantly impact forecasting accuracy. A more comprehensive analysis of
these factors could provide a fuller understanding of the uncertainties involved
in transportation demand modelling.

\subsubsection{Static assignment constraint}
\RC The research suggests that the static network assignment may constrain the
possible volume solutions, potentially limiting the practical impacts of
parameter uncertainty. This raises questions about the generalizability of the
findings to more dynamic or less constrained network scenarios.

\subsubsection{Trip-based model}
\RC The findings are based on a specific trip-based travel demand model, which
may not be applicable to all contexts or regions. The results might vary
significantly in different geographic areas or under different modelling
frameworks, such as activity-based models.


\end{document}
