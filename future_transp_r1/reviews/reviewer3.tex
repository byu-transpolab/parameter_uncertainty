% An example for the ar2rc document class.
% Copyright (C) 2017 Martin Schroen
% Modifications Copyright (C) 2020 Kaishuo Zhang
%
% This program is free software: you can redistribute it and/or modify
% it under the terms of the GNU General Public License as published by
% the Free Software Foundation, either version 3 of the License, or
% (at your option) any later version.
%
% This program is distributed in the hope that it will be useful,
% but WITHOUT ANY WARRANTY; without even the implied warranty of
% MERCHANTABILITY or FITNESS FOR A PARTICULAR PURPOSE.  See the
% GNU General Public License for more details.
%
% You should have received a copy of the GNU General Public License
% along with this program.  If not, see <http://www.gnu.org/licenses/>.

\documentclass{ar2rc}

\title{Evaluating the Impacts of Parameter Uncertainty in a Practical Transportation Demand Model}
\author{G. Macfarlane}
\journal{Future Transportation}
\doi{futuretransp-3339770}

\begin{document}

\maketitle

We are grateful for the reviewer's comments on this manuscript. We address each point in turn, with 
\DIFaddbegin \DIFadd{text added to the manuscript in blue}\DIFaddend  and 
\DIFdelbegin \DIFdel{text deleted from the manuscript in red}\DIFdelend.

\section{Reviewer \#3}

\subsection{General Comments}
\RC The paper is well-structured and follows a logical progression from introduction to conclusion. The
research addresses an important issue in transportation demand modeling—parameter uncertainty. It
adds value to the field by analyzing its impacts using Latin Hypercube Sampling (LHS). 

\AR We appreciate that the reviewer found our paper well-structured and on an important issue.

\RC However, the real-world implications of the findings could be emphasized
more, particularly in the "Conclusions" section.

\AR We address this comment further below when it arises again.

\subsection{Keywords}
\RC Keywords: Having only two keywords is insufficient. Consider expanding the list to include
more relevant terms to improve discoverability and accurately reflect the scope of the
research.

\subsection{Travel modeling key challenges}
\RC Line 21: Before introducing the function in this line, provide a more comprehensive overview of
the current state of transportation demand modeling and its key challenges. This will help set
the context for your research.

\subsection{Literature Review}
\RC Line 57: For the literature review, it may be more effective to organize the discussion by related
topics or themes rather than describing each study in detail.

\subsection{Number of draws}
\RC Line 293: The rationale for selecting 100 and 600 draws needs to be explained more clearly.
Why were these specific numbers chosen? Discuss how they are sufficient for capturing the
parameter variability in this study.

\subsection{Figure placement}
\RC Figures 5 and 6: These figures should be placed before the conclusion section to ensure that all
supporting visualizations are discussed prior to summarizing the findings.

\AR We are using the \LaTeX formatting engine, which may move floating objects forward or backward 
depending on available page space; we leave this to editorial choice.

\subsection{Practical impacts}
\RC Conclusion: The conclusion section needs to be strengthened. Summarize the main results
clearly and discuss their implications for real-world applications. Additionally, outline potential
directions for future research to provide a comprehensive closure.

\end{document}
