% An example for the ar2rc document class.
% Copyright (C) 2017 Martin Schroen
% Modifications Copyright (C) 2020 Kaishuo Zhang
%
% This program is free software: you can redistribute it and/or modify
% it under the terms of the GNU General Public License as published by
% the Free Software Foundation, either version 3 of the License, or
% (at your option) any later version.
%
% This program is distributed in the hope that it will be useful,
% but WITHOUT ANY WARRANTY; without even the implied warranty of
% MERCHANTABILITY or FITNESS FOR A PARTICULAR PURPOSE.  See the
% GNU General Public License for more details.
%
% You should have received a copy of the GNU General Public License
% along with this program.  If not, see <http://www.gnu.org/licenses/>.

\documentclass{ar2rc}

\title{Evaluating the Impacts of Parameter Uncertainty in a Practical Transportation Demand Model}
\author{G. Macfarlane}
\journal{Future Transportation}
\doi{futuretransp-3339770}

\begin{document}

\maketitle

We are grateful for the reviewer's comments on this manuscript. We address each point in turn, with 
\DIFaddbegin \DIFadd{text added to the manuscript in blue}\DIFaddend  and 
\DIFdelbegin \DIFdel{text deleted from the manuscript in red}\DIFdelend.

\section{Reviewer \#3}

\subsection{General Comments}
\RC The paper is well-structured and follows a logical progression from introduction to conclusion. The
research addresses an important issue in transportation demand modeling—parameter uncertainty. It
adds value to the field by analyzing its impacts using Latin Hypercube Sampling (LHS). 

\AR We appreciate that the reviewer found our paper well-structured and on an important issue.

\RC However, the real-world implications of the findings could be emphasized
more, particularly in the "Conclusions" section.

\RC [moved from end of review] Conclusion: The conclusion section needs to be strengthened. Summarize the main results
clearly and discuss their implications for real-world applications. Additionally, outline potential
directions for future research to provide a comprehensive closure.

\AR We weclome the invitation to strengthen our conclusions, and have done so in a revised conclusions section.

\begin{quote}
    
\DIFdelbegin \DIFdel{In general, this research has shown that statistical parameter uncertainty does not appear to be a significant factor in forecasting
traffic volumes using trip-based travel demand models.
The result
uncertainty is generally equal to or }\DIFdelend \DIFaddbegin \DIFadd{The results of this research show that despite large variations in 
mode and destination choice parameters – and consequently large variations
in accessibility, the impact of this variation on assigned highway volumes
is limited. To our knowledge, this is the first systematic evaluation of 
parameter uncertainty in a practical travel model in the literature, with prior
research being limited to toy networks (e.g., Zhao and Kockelman, 2002).
The resulting uncertainty in the output forecasts was shown to be generally }\DIFaddend smaller 
than the input parameter variance\DIFdelbegin \DIFdel{. The uncertainty in parameter inputs appears to lead to
variation in highway volumes that is lower than the error between the
model forecast and the highway counts.
Any }\DIFdelend \DIFaddbegin \DIFadd{, confirming the results of Petrik et al.
(2020) in a different context.
In this application at least, the }\DIFaddend variation in mode and
destination choice probabilities appears to be constrained by the
\DIFdelbegin \DIFdel{limitations }\DIFdelend \DIFaddbegin \DIFadd{capacities and procedures }\DIFaddend of the highway network assignment.

There are several limitations that must be mentioned in this research,
however. First, we did not attempt to address the statistical
uncertainty in trip production estimates; these may play a substantially
larger role than destination and mode choice parameters, given that
lower trip rates may lead to lower traffic volumes globally, which could
not be ``corrected'' by the static user equilibrium assignment. \DIFaddbegin \DIFadd{Second, a different
methodology of sampling might have produced a different result at the extremes than
the results of LHS.
}\DIFaddend Additionally, the relatively sparse network of the RVTPO model region
--- lacking parallel high-capacity highway facilities --- may have meant
that the static network assignment would converge to a similar solution
point regardless of modest changes to the trip matrix. It may be that in
a larger network with more path redundancies \DIFdelbegin \DIFdel{, }\DIFdelend \DIFaddbegin \DIFadd{or more alternative transit services
}\DIFaddend the assignment may not have been as helpful in constraining the forecast
volumes.

In this research we had only the estimates of the statistical
coefficients, and therefore had to assume a coefficient of variation to
derive variation in the sampling procedure. It would be better if model
user and development documentation more regularly provided estimates of
the standard errors of model parameters. \DIFdelbegin \DIFdel{Even better }\DIFdelend \DIFaddbegin \DIFadd{The ideal }\DIFaddend would be
variance-covariance matrices for the estimated models, enabling
researchers to ensure that covariance relationships between sampled
parameters are maintained. \DIFaddbegin \DIFadd{Future research might reconsider the present 
expermiment but allowing for correlation between parameter values.
}\DIFaddend 
\end{quote}


\subsection{Keywords}
\RC Keywords: Having only two keywords is insufficient. Consider expanding the list to include
more relevant terms to improve discoverability and accurately reflect the scope of the
research.

\AR Thank you for this suggestion. We have expanded with two additional keywords,
\begin{quote}
    Travel modeling; Uncertainty; \DIFaddbegin \DIFadd{Choice models; Trip-based models}\DIFaddend 
\end{quote}

\subsection{Travel modeling key challenges}
\RC Line 21: Before introducing the function in this line, provide a more comprehensive overview of
the current state of transportation demand modeling and its key challenges. This will help set
the context for your research.

\AR This is a good point. We have strengthened the first paragraph of the
manuscript by including references to the attention uncertainy is receiving in the 
travel forecasting community,

\begin{quote}
    The inherent accuracy and uncertainty in travel forecasting models is
    receiving increasing attention from the scholarly and practicing
    community. \DIFaddbegin \DIFadd{As an example of this attention, the Standing Committee on 
    Transport Forecasting of the Transportation Research Board has made uncertainty
    one of its primary research agenda issues (TRB AEP50, 2025), following a major report from 
    Federal Highway Administration on the topic (Lempert et al., 2022).
    }\DIFaddend 
\end{quote}

\subsection{Literature Review}
\RC Line 57: For the literature review, it may be more effective to organize the discussion by related
topics or themes rather than describing each study in detail.

\AR While we agree that the literature review could be more coherently organized, the time necessary
to do this revision has not been granted by the journal's editorial staff.

\subsection{Number of draws}
\RC Line 293: The rationale for selecting 100 and 600 draws needs to be explained more clearly.
Why were these specific numbers chosen? Discuss how they are sufficient for capturing the
parameter variability in this study.

\AR We agree that this could have used more explanation. We have revised this
section extensively to improve clarity.

\begin{quote}
    ith the trip-based model described above, MC and LHS methods were used
    to develop alternative parameter sets to evaluate uncertainty. To
    identify a standard deviation for each parameter, \DIFaddbegin \DIFadd{we asserted }\DIFaddend a coefficient of
    variation \DIFdelbegin \DIFdel{was used. A set coefficient of variation of 0.10 was used for
    }\DIFdelend \DIFaddbegin \DIFadd{\(c_v = 0.10 \) }\DIFaddend the four mode choice coefficients and the destination
    choice parameters\DIFdelbegin \DIFdel{.
    The }\DIFdelend \DIFaddbegin \DIFadd{; the }\DIFaddend mode choice constants \DIFdelbegin \DIFdel{were kept the same }\DIFdelend \DIFaddbegin \DIFadd{remained fixed }\DIFaddend across all iterations.
    Literature had identified a coefficient of variation of 0.30, but for
    this analysis that caused an unrealistic value of time, and thus it was
    changed to be 0.10 (Zhao and Kockelman, 2002). Value of time is a ratio
    in units of money per time that should be compared to the regional wage
    rate. \DIFdelbegin \DIFdel{Using a }\DIFdelend \DIFaddbegin \DIFadd{A }\DIFaddend \(c_v\) of 0.30 \DIFdelbegin \DIFdel{the }\DIFdelend \DIFaddbegin \DIFadd{implied a }\DIFaddend value of time \DIFdelbegin \DIFdel{range was }\DIFdelend \DIFaddbegin \DIFadd{extending }\DIFaddend from \(\$2\)
    to \(\$32\) \DIFdelbegin \DIFdel{/hr, whereas using }\DIFdelend \DIFaddbegin \DIFadd{per hour, whereas }\DIFaddend a \(c_v\) of 0.10 \DIFdelbegin \DIFdel{the range was }\DIFdelend \DIFaddbegin \DIFadd{implied values between }\DIFaddend \(\$6\)
    to \(\$14\) /hr\DIFdelbegin \DIFdel{. The latter seemed more rational because it is related
    to wage rates and thus a \(c_v\) of 0.10 was used for our analysis}\DIFdelend \DIFaddbegin \DIFadd{, which we assess as more reasonable for this context}\DIFaddend .
    The standard deviation \DIFdelbegin \DIFdel{was }\DIFdelend \DIFaddbegin \DIFadd{for sampling the parameters was therefore }\DIFaddend equal to 0.10
    multiplied by the mean, where the mean values in this situation are the base
    scenario parameters (as identified in Table 2 ).
    
    The MC random sampling uses the R function of \texttt{rnorm}. LHS uses
    the \texttt{lhs} package in R.  \DIFdelbegin \DIFdel{Since this package only chooses variables
    on a zero to one scale, the values given use a function to put the
    random sampling on the right scale needed for the given parameter. }\DIFdelend The full code for both methods can be found in a
    public \href{https://github.com/natmaegray/sensitivity_thesis}{GitHub
    repository}. \DIFdelbegin \DIFdel{One hundred and }\DIFdelend \DIFaddbegin \DIFadd{We wish to ensure in our simulations first, that we explore the
    full parameter uncertainty space of the model, and second that we run a
    sufficient number of simulations that outlying and extreme draws do not overly
    influence our analysis. We therefore designed and present a short experiment to
    evaluate the average mode choice logsum in the model determined by 100 and }\DIFaddend 600
    \DIFdelbegin \DIFdel{draws of random samples for both
    methods are generated. With these generated parameters, the mode choice
    modelstep was run for every set of input parameters for each purpose. The average MCLS value for each run was determined to compare each
    continuous draw. This allowed us to see how many iterations of which
    sampling type would be sufficient to show a full range of possible
    outcomes.
    }\DIFdelend \DIFaddbegin \DIFadd{draws of parameters via both MC and LHS.
    }\DIFaddend 
    
    \DIFdelbegin \DIFdel{The parameters generated were compared for both sampling methods.
    }\DIFdelend Figure~1 shows the distributions \DIFdelbegin \DIFdel{for }\DIFdelend \DIFaddbegin \DIFadd{of }\DIFaddend the HBW
    parameters when using 100 and 600 draws\DIFaddbegin \DIFadd{, including the distribution of implied 
    value of time, which is an indirect number based on two separate draws}\DIFaddend .
    These distributions show \DIFaddbegin \DIFadd{in general }\DIFaddend that LHS gives normally distributed parameters with fewer draws than MC
    sampling\DIFdelbegin \DIFdel{: at 100 draws LHS shows a nearly perfect normal distribution, where there are some discrepancies for the MC generated parameters.
    These Figures show that LHS is likely to estimate the full variance of
    the results with fewer draws.
    }\DIFdelend \DIFaddbegin \DIFadd{, as expected by theory (Helton and Davis, 2003).
    }\DIFaddend 

    \setcounter{equation}{4}
    To determine if LHS is effective at a reasonable amount of iterations,
the cumulative mean and the cumulative standard deviation of the \DIFdelbegin \DIFdel{average
}\DIFdelend \DIFaddbegin \DIFadd{mean 
}\DIFaddend MCLS value for \DIFdelbegin \DIFdel{every zone }\DIFdelend \DIFaddbegin \DIFadd{all zones }\DIFaddend (see Equation~2 ) was calculated
for each additional draw for both sampling methods. MCLS is an impedance
term which is an important value for destination choice and region
routing. The \DIFdelbegin \DIFdel{average }\DIFdelend \DIFaddbegin \DIFadd{mean }\DIFaddend MCLS, \(x\), was used as a measure of outcome
possibilities to simplify a complex term as a single value to compare by
across all iterations. The cumulative mean is calculated as:
\begin{equation}\phantomsection\label{eq-cmclsmean}{
\mu_i = \frac{x_1 + ... + x_i}{n}
}\end{equation} and the cumulative standard deviation is calculated as:
\begin{equation}\phantomsection\label{eq-sdi}{
SD_i = \sqrt{\frac{\sum (x_i - \mu_i)^2 }{n-1}}.
}\end{equation}  \DIFdelbegin \DIFdel{The cumulative mean shows }\DIFdelend \DIFaddbegin \DIFadd{Figure~2 illustrates }\DIFaddend how the average MCLS
stabilizes \DIFdelbegin \DIFdel{across each iteration}\DIFdelend \DIFaddbegin \DIFadd{as the number of draws increases}\DIFaddend , and the cumulative standard deviation
is used to show the 95\% confidence interval of that mean. 
\DIFdelbegin \DIFdel{When the
cumulative mean for the draws stabilizes, that shows that the amount of
generated parameters has captured the possible variance of the results.
This is shown for two of the three trip purposes in Figure~2.
}\DIFdelend 

    
\end{quote}

\subsection{Figure placement}
\RC Figures 5 and 6: These figures should be placed before the conclusion section to ensure that all
supporting visualizations are discussed prior to summarizing the findings.

\AR We are using the \LaTeX\  formatting engine, which may move floating objects forward or backward 
depending on available page space; we leave this to editorial choice.





\end{document}
