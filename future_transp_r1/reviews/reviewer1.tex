% An example for the ar2rc document class.
% Copyright (C) 2017 Martin Schroen
% Modifications Copyright (C) 2020 Kaishuo Zhang
%
% This program is free software: you can redistribute it and/or modify
% it under the terms of the GNU General Public License as published by
% the Free Software Foundation, either version 3 of the License, or
% (at your option) any later version.
%
% This program is distributed in the hope that it will be useful,
% but WITHOUT ANY WARRANTY; without even the implied warranty of
% MERCHANTABILITY or FITNESS FOR A PARTICULAR PURPOSE.  See the
% GNU General Public License for more details.
%
% You should have received a copy of the GNU General Public License
% along with this program.  If not, see <http://www.gnu.org/licenses/>.

\documentclass{ar2rc}

\title{Evaluating the Impacts of Parameter Uncertainty in a Practical Transportation Demand Model}
\author{G. Macfarlane}
\journal{Future Transportation}
\doi{futuretransp-3339770}

\begin{document}

\maketitle

\section{Reviewer \#1}

\subsection{Introduction}
\RC Introduction should more clearly state objectives of this study, especially
the specific issues regarding impacts of parameter uncertainty in traffic demand
model on prediction results. The current description is a bit vague, and it is
recommended to add more explanations on the importance and practical
significance of the study.

\AR We have added the following sentences to the Introduction to clarify the relationship
between uncertainty and prediction

\begin{quote}
    \DIFaddbegin \DIFadd{Systemic under- or over-prediction could lead to substantial
    over- or under- investment in the highway network (Hoque et al., 2021).
    }\DIFaddend 

    \ldots

    A detailed description of specifying mode choice model variables and nesting
    of error structures is given by Koppelman and Bhat (2006)\DIFaddbegin
    \DIFadd{. This category includes deeper uncertainty issues such as unforseen
    shifts in behavior}\DIFaddend .
\end{quote}

\subsection{Application Context}
\RC In literature review, authors are advised to cite relevant literature to
expand the application context of traffic model. For example,
doi.org/10.1016/j.physa.2024.12954.

\AR The DOI link supplied by the reviewer was malformed, and we are therefore
unable to respond to this comment.

\subsection{Model Design}
\RC In Model Design and Methodology, author is advised to provide more detailed
model construction process, assumptions, and specific application of the
selected method (such as LHS).

\AR This is not a clear comment to us. We have provided a detailed description
of approximately 1400 words with an accompanying full-source GitHub repository;
and this comment is not specific enough to lead to a response action. We did, however, 
note and correct a typo in an equation reference in this section.

\begin{quote}
 These PA trips are converted into origin
destination (OD) trips by multiplying the trips by corresponding time of
day factors (see \DIFdelbegin \DIFdel{\#eq-trips}\DIFdelend \DIFaddbegin \DIFadd{Equation 1}\DIFaddend ). These trips are calculated using Bentley's
CUBE and the RVTPO model.
\end{quote}


\subsection{Sampling Methodology}
\RC The authors selected the LHS and MC methods for uncertainty design, but did
not describe the rationale and advantages of selecting these methods. It is
recommended that the authors add a description of the study method selection
process, including the consideration and rationale for the exclusion of other
potential methods.

\RC This is a reasonable point. We have addressed it in the Methodology with an additional
reference,

\begin{quote}
    Two common methods for parameter sampling include \DIFdelbegin \DIFdel{, }\DIFdelend Monte
Carlo (MC) simulation and Latin hypercube sampling (LHS)\DIFdelbegin \DIFdel{. }\DIFdelend \DIFaddbegin \DIFadd{;  
In general, }\DIFaddend MC simulation
draws independently from multiple distributions, while LHS makes draws
that cover the parameter space more efficiently and can capture the
joint distribution between two or more parameter values (Helton and
Davis, 2003). As a result, LHS can reduce the number of draws needed to
fully re-create the statistical variance in a model, but the amount of
reduction is unknown and may not be universal to all problems (Yang et
al., 2013).  \DIFaddbegin \DIFadd{And though more potential
methods are being developed and employed in related research (e.g., Singh et al., 2024) 
this research only considers these two.
}\DIFaddend 


\DIFadd{Singh, A., Mondal, S., Pandey, R., Jha, S. K. (2024). Assessing Fourier and
Latin hypercube sampling methods as new multi-model methods for hydrological
simulations. Stochastic Environmental Research and Risk Assessment, 38(4),
1271–1295. }
\DIFaddend 
\end{quote}

\RC and similarly in the limitations,

\begin{quote}
    
    \DIFaddbegin \DIFadd{Second, a different
    methodology of sampling might have produced a different result at the extremes than
    the results of LHS.
    }\DIFaddend 
\end{quote}


\subsection{Data collection}
\RC The source and preprocessing process of the data used are not clearly stated
in this paper. It is recommended to add a detailed description of the data
collection, cleaning, and processing methods.

\AR We did not collect, clean, or process any data for this paper, and therefore the 
the paper does not describe these processes. Rather, the paper generates its own data 
Monte Carlo and Latin Hypercube Sampling, as described.

\end{document}
